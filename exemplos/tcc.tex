\documentclass{abntex2}

% essa linha eh um comentario!

\usepackage[utf8]{inputenc}  % acentuacao

\usepackage{lipsum}  % gerador de lero-lero

\usepackage{amsthm}  % coisas matematicas
\newtheorem{definicao}{Definição}[section]  % ambiente definicao
\newtheorem{teorema}{Teorema}[chapter]  % ambiente teorema

\author{Adriano}
\title{Meu querido TCC}

\begin{document}
\maketitle

\chapter{Introdução}
\lipsum[1-8]

\chapter{Matemática: diversão!}
$a^2=b^2+c^2$

$$a^2=b^2+c^2$$

\begin{equation}
	a^2=b^2+c^2
	\label{pitagoras}
\end{equation}

\begin{definicao}
	Uma antiderivada de uma função $f$ em $I$ é uma função $F$ tal que $F'(x)=f(x),\forall x\in I$.
\end{definicao}

\begin{teorema}(Teorema Fundamental do Cálculo)
	\[\int_a^b f(x) dx = F(b)-F(a).\]
	\label{tfc}
\end{teorema}
\begin{proof}
	\lipsum[1]
\end{proof}

Pelo Teorema \ref{tfc} e por Pitágoras (equação \ref{pitagoras}), segue o resultado... 

\begin{teorema}
	Esse é mais um teorema...
\end{teorema}



\end{document}
